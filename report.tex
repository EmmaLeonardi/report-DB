\documentclass{report}
\usepackage[utf8]{inputenc}
\usepackage{url}

\title{Elaborato per il corso Basi di Dati\\ \small Progetto di una base di dati per la gestione di un sito di e-Commerce}

\author{
    Studente: Emma Leonardi\\
    Email: \url{emma.leonardi2@studio.unibo.it}\\
    Matricola: 0000971438\\
    A.A. 2021/2022
    }
\date{}
\begin{document}

\maketitle
\tableofcontents

\chapter{Analisi}
\section{Analisi dei requisiti}
Descrizione requisiti
\section{Intervista}
Primo testo, esempi 
\section{Rilevamento delle ambiguità e correzioni proposte}
Correzoni
\section{Definizione delle specifiche in linguaggio naturale ed estrazione dei concetti principali}
Estrazione concetti\\
Tabella termini-> concetto, correzione

\chapter{Progettazione Concettuale}
\section{Schema scheletro}
ER, ragionamenti su ER, immagini ER
\section{Raffinamenti proposti}
ragionamenti su ER
\section{Schema concettuale finale}
ER finale

\chapter{Progettazione Logica}
\section{Stima del volume dei dati}
Tabella volume
\section{Descrizione delle operazioni principali e stima della loro frequenza}
Tabella op frequenza
\section{Schemi di navigazione e tabelle degli accessi}
Conti pesantezza operazioni, con tabelle
\section{Raffinamento dello schema}
eliminazione di identificatori esterni, attributi composti e gerarchie, scelta delle chiavi
subsection per ognuno di questi lavori
\section{Analisi delle ridondanze}
Ridondanze, conti
\section{Traduzione di entità e associazioni in relazioni}
Traduzione ER
\section{Schema relazionale finale}
Schema finale
\section{Traduzione delle operazioni in query SQL}
delle operazioni

\chapter{Progettazione dell'applicazione}
\section{Descrizione dell'architettura dell'applicazione}
Java, operazioni banali (inserimento, ecc)
realizzata con obbligo di inserire alcuni screenshot dell'interfaccia utente
diverse schermate (utente, corriere, ecc)
tabella operazione originale -> metodo/classe
\end{document}