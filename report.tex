\documentclass{report}
\usepackage[utf8]{inputenc}
\usepackage{url}
\usepackage[italian]{babel}
\usepackage[italian]{cleveref}

\title{Elaborato per il corso Basi di Dati\\ \small Progetto di una base di dati per la gestione di un sito di e-Commerce}

\author{
    Studente: Emma Leonardi\\
    Email: \url{emma.leonardi2@studio.unibo.it}\\
    Matricola: 0000971438\\
    A.A. 2021/2022
    }
\date{}
\begin{document}

\maketitle
\tableofcontents

\chapter{Analisi}
\section{Analisi dei requisiti}
Si ha intenzione di creare un database per gestire un sito di e-Commerce.
Il database dovrà memorizzare informazioni sui prodotti in vendita, gestire l'interazione con il cliente che compra i prodotti. 
Inoltre dovrà gestire anche i corrieri che effettuano le consegne delle spese effettuate e sapere in quale fabbrica viene prodotto quale prodotto.
\section{Intervista}
Si vuole tenere traccia degli acquisti dei clienti. I dati dei clienti memorizzati sono nome, cognome, codice fiscale, data di nascita, email e numero di telefono, coordinate bancarie e indirizzo di residenza.
Il cliente acquista una spesa con un costo che è la somma dei costi dei prodotti presenti nella spesa. 
I prodotti possono cambiare prezzo nel tempo e bisogna tenere memorizzato lo storico. Dei prodotti si salva il materiale, la descrizione, la taglia se abbigliamento o la scadenza se prodotto alimentare. 
I prodotti vengono creati in fabbriche, gestite da più produttori in periodi diversi. Del produttore si salva la partita IVA e della fabbrica l'indirizzo. 
La spesa viene consegnata da un corriere e si salva la data. Un corriere guida un mezzo, di cui si memorizzano il tipo di veicolo, la marca, il paese di immatricolazione e la targa.
Del corriere vengono memorizzate la nazionalità della patente, il codice e i turni di guida. La consegna può essere standard o premium, cambia il prezzo della consegna.
La consegna ha collegato un indirizzo, che non per forza è l'indirizzo di residenza del cliente. Più corrieri non possono guidare lo stesso mezzo in contemporanea, la consegna è effettuata da un unico corriere 
e si riferisce ad una sola spesa. Una spesa è riferita a un solo cliente, ma un cliente può effettuare più spese. Una fabbrica viene gestita da un solo produttore alla volta e può produrre diversi prodotti. 
Un prodotto viene fabbricato in una sola fabbrica. Si memorizzano nome, cognome, codice fiscale e data di nascita per i clienti, produttori e corrieri.


\section{Rilevamento delle ambiguità e correzioni proposte}
Correzoni
\section{Definizione delle specifiche in linguaggio naturale ed estrazione dei concetti principali}
Estrazione concetti\\
Tabella termini-> concetto, correzione

\chapter{Progettazione Concettuale}
\section{Schema scheletro}
ER, ragionamenti su ER, immagini ER
\section{Raffinamenti proposti}
ragionamenti su ER
\section{Schema concettuale finale}
ER finale

\chapter{Progettazione Logica}
\section{Stima del volume dei dati}
Tabella volume
\section{Descrizione delle operazioni principali e stima della loro frequenza}
Tabella op frequenza
\section{Schemi di navigazione e tabelle degli accessi}
Conti pesantezza operazioni, con tabelle
\section{Raffinamento dello schema}
eliminazione di identificatori esterni, attributi composti e gerarchie, scelta delle chiavi
subsection per ognuno di questi lavori
\section{Analisi delle ridondanze}
Ridondanze, conti
\section{Traduzione di entità e associazioni in relazioni}
Traduzione ER
\section{Schema relazionale finale}
Schema finale
\section{Traduzione delle operazioni in query SQL}
delle operazioni

\chapter{Progettazione dell'applicazione}
\section{Descrizione dell'architettura dell'applicazione}
Java, operazioni banali (inserimento, ecc)
realizzata con obbligo di inserire alcuni screenshot dell'interfaccia utente
diverse schermate (utente, corriere, ecc)
tabella operazione originale -> metodo/classe
\end{document}